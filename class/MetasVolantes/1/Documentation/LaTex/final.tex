\documentclass[conference]{IEEEtran}
\IEEEoverridecommandlockouts
% The preceding line is only needed to identify funding in the first footnote. If that is unneeded, please comment it out.
\usepackage{cite}
\usepackage{amsmath,amssymb,amsfonts}
\usepackage{algorithmic}
\usepackage{graphicx}
\usepackage{textcomp}
\usepackage{xcolor}
\usepackage{multirow}
\def\BibTeX{{\rm B\kern-.05em{\sc i\kern-.025em b}\kern-.08em
    T\kern-.1667em\lower.7ex\hbox{E}\kern-.125emX}}
\begin{document}

\title{Meta Volante 1}

\author{\IEEEauthorblockN{David Calle Gonzalez}
\IEEEauthorblockA{\textit{dept. of science} \\
\textit{EAFIT}\\
Medellín, Colombia \\
dcalleg@eafit.edu.co}
\and
\IEEEauthorblockN{Santiago Gil Zapata}
\IEEEauthorblockA{\textit{dept. of science} \\
\textit{EAFIT}\\
Medellín, Colombia \\
sgilz@eafit.edu.co}
\and
\IEEEauthorblockN{Sebastian Obando}
\IEEEauthorblockA{\textit{dept. of science} \\
\textit{EAM}\\
Medellín, Colombia \\
sebastian.obando.8888@eam.edu.co}
\and
\IEEEauthorblockN{Juan Manuel Young Hoyos}
\IEEEauthorblockA{\textit{dept. of science} \\
\textit{EAFIT}\\
Medellín, Colombia \\
jmyoungh@eafit.edu.co}
}

\maketitle

\begin{abstract}
El presente documento tiene por objetivo demostrarlos procesos necesarios para poder ejecutar HPL en
nuestro cluster de 2 nodos en Cronos. 
\end{abstract}

\begin{IEEEkeywords}
HPC, HPL, MPI.
\end{IEEEkeywords}

\section{Introduction}
La idea del proyecto es mostrar cómo logramos una eficiencia de \( X\% \).

\section{Objectives}
Lograr una eficiencia entre \( X\% \) y \( Y\% \) usando solo los 2 nodos de Cronos que tenemos
a nuestra disposición.

\section{Execution environment}

En este apartado se mostrará qué se ha usado para realizar estas pruebas.

\subsection{Hardware}

\begin{itemize}
    \item \textbf{Cantidad de nodos:} 2.
    \item \textbf{Procesadores por nodo:} Intel Xeon E5-2670 0 (16) @ 2.989GHz.
    \item \textbf{Controlador Ethernet:} Intel Corporation I350 Gi-gabit Network Connection.
    \item \textbf{Memoria por nodo:} 16 x 4GB DIMM DDR3 1333MT/s.
\end{itemize}

\subsection{Software}

\begin{itemize}
    \item \textbf{HPL:} 2.3 \cite{1}.
    \item \textbf{Sistema Operativo:} CentOS Linux 8 (Core) \(x86_64\).
    \item \textbf{BLAS:} 3.10.0 \cite{2}.
    \item \textbf{MPI:} icc (ICC) 2021.2.0 20210228.
    \item \textbf{Compiler:} icc (ICC) 2021.2.0 20210228.
    \item \textbf{NFSv3}.
\end{itemize}

\section{HPL Heaven}
La definición de que tenemos de este concepto abarca varios razonamientos
que hace complicado dar una definición concreta de este mismo. De una forma 
más acertada y general podríamos decir que "Heaven" hace referencia a el valor teórico 
más optimo al que podríamos llegar; sin embargo, este valor en la práctica es imposible 
de alcanzar por lo que siempre intentamos tener algo lo más cercano porsible a este mismo.

\section{HPL Optimizations}

\subsection{HPL.dat}
Los valores de parametros que se encuentran definidos en \textit{HPL.dat} tiene
un alto impacto en el performace del mismo, especialmente 3 de estos que son:

\begin{enumerate}
    \item El tamaño de la matriz de nuestro problema (N): Este valor debe seleccionarse basados
    en el sistema de hardware y software que se tenga; Se debe decidir teniendo en cuenta la 
    eficiencia computacional y la capacidad en memoria. A mayor el tamaño de la matriz, mayor 
    serán los Flops.
    
    \item El tamaño del bloque (NB): Este valor es usualmente determinado por medio de experimentación.
    Sin embargo, debe de ser cuidadosamente seleccionado para no ser ni muy grande ni muy pequeño, suelen 
    ser números menores a 512 y normalmente multiplos de 64.
    
    \item La matriz de procesos bidimencional (P X Q): El resultado del producto debe ser equivalente al total
    del número de procesos asignados en el parametro -n al ejecutar hpl. Como recomendación general se tendría
    a P < Q. Donde P debería de tomar el valor más pequeño que sea posible.
\end{enumerate}
    

    
\section{Conclusion}

Conclusión

\begin{thebibliography}{00}
\bibitem{1} J. D. A. C. P. L. Antoine Petitet, Clint Whaley, “Hpl 2.3 - aportable implementation of the high-performance linpack.” [Online]. Available at: http://www.netlib.org/benchmark/hpl/software.html.
\bibitem{2} [Online]. Available at: http://www.netlib.org/blas/.
\end{thebibliography}

\end{document}
