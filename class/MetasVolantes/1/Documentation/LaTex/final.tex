\documentclass[conference]{IEEEtran}
\IEEEoverridecommandlockouts
% The preceding line is only needed to identify funding in the first footnote. If that is unneeded, please comment it out.
\usepackage{cite}
\usepackage{amsmath,amssymb,amsfonts}
\usepackage{algorithmic}
\usepackage{graphicx}
\usepackage{textcomp}
\usepackage{xcolor}
\usepackage{multirow}
\def\BibTeX{{\rm B\kern-.05em{\sc i\kern-.025em b}\kern-.08em
    T\kern-.1667em\lower.7ex\hbox{E}\kern-.125emX}}
\begin{document}

\title{Meta Volante 1}

\author{\IEEEauthorblockN{David Calle Gonzalez}
\IEEEauthorblockA{\textit{dept. of science} \\
\textit{EAFIT}\\
Medellín, Colombia \\
dcalleg@eafit.edu.co}
\and
\IEEEauthorblockN{Santiago Gil Zapata}
\IEEEauthorblockA{\textit{dept. of science} \\
\textit{EAFIT}\\
Medellín, Colombia \\
sgilz@eafit.edu.co}
\and
\IEEEauthorblockN{Sebastian Obando}
\IEEEauthorblockA{\textit{dept. of science} \\
\textit{EAM}\\
Medellín, Colombia \\
sebastian.obando.8888@eam.edu.co}
\and
\IEEEauthorblockN{Juan Manuel Young Hoyos}
\IEEEauthorblockA{\textit{dept. of science} \\
\textit{EAFIT}\\
Medellín, Colombia \\
jmyoungh@eafit.edu.co}
}

\maketitle

\begin{abstract}
El presente documento tiene por objetivo demostrarlos procesos necesarios para poder ejecutar HPL en
nuestro cluster de 2 nodos en Cronos. 
\end{abstract}

\begin{IEEEkeywords}
HPC, HPL, MPI.
\end{IEEEkeywords}

\section{Introduction}
La idea del proyecto es mostrar cómo logramos una eficiencia de \( X\% \).

\section{Objectives}
Lograr una eficiencia entre \( X\% \) y \( Y\% \) usando solo los 2 nodos de Cronos que tenemos
a nuestra disposición.

\section{Execution environment}

En este apartado se mostrará qué se ha usado para realizar estas pruebas.

\subsection{Hardware}

\begin{itemize}
    \item \textbf{Cantidad de nodos:} 2.
    \item \textbf{Sistema Operativo:} CentOS Linux 8 (Core) \(x86_64\).
    \item \textbf{Procesadores por nodo:} Intel Xeon E5-2670 0 (16) @ 2.989GHz.
    \item \textbf{Controlador Ethernet:} Intel Corporation I350 Gi-gabit Network Connection.
    \item \textbf{Memoria por nodo:} 16 x 4GB DIMM DDR3 1333MT/s.
\end{itemize}

\subsection{Software}

\begin{itemize}
    \item \textbf{HPL:} 2.3 \cite{1}.
    \item \textbf{Sistema Operativo:} CentOS Linux 8 (Core) \(x86_64\).
    \item \textbf{Procesadores por nodo:} Intel Xeon E5-2670 0 (16) @ 2.989GHz.
    \item \textbf{Controlador Ethernet:} Intel Corporation I350 Gi-gabit Network Connection.
    \item \textbf{Memoria por nodo:} 16 x 4GB DIMM DDR3 1333MT/s.
\end{itemize}

\section{Conclusion}

Conclusión

\begin{thebibliography}{00}
\bibitem{1} J. D. A. C. P. L. Antoine Petitet, Clint Whaley, “Hpl 2.3 - aportable implementation of the high-performance linpack.” [Online]. Available at: http://www.netlib.org/benchmark/hpl/software.html.
\bibitem{2} Chapra, S. and Canale, R., 2003. Numerical methods for engineers. Boston: McGraw-Hill.
\bibitem{3}L. Dalcin, et al.,"Cython: The Best of Both Worlds" in Computing in Science  Engineering, vol. 13, no. 02, pp. 31-39, 2011.doi: 10.1109/MCSE2010.118
\bibitem{4} A Brief Description - C++ Information", Cplusplus.com, 2021. [Online]. Available: https://www.cplusplus.com/info/description/.
\bibitem{5} Python vs C++ Comparison: Compare Python vs C++ Speed and More", BitDegree.org Online Learning Platforms, 2021. [Online]. Available: https://www.bitdegree.org/tutorials/python-vs-c-plus-plus/.
\end{thebibliography}

\end{document}
